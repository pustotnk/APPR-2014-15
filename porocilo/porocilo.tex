\documentclass[11pt,a4paper]{article}
\usepackage[slovene]{babel}
\usepackage[utf8x]{inputenc}
\usepackage{graphicx}
\usepackage{url}
\usepackage{hyperref}
\usepackage{pdfpages}
\pagestyle{plain}
\begin{document}
\title{Poročilo pri predmetu \\
Analiza podatkov s programom R}
\author{Tomaž Pustotnik}
\maketitle

\section{Izbira teme}

Za tematiko moje seminarske naloge, sem si kot navijač nogometnega kluba FC Barcelona izbral analizo podatkov najboljših štirideset strelcev vseh časov, ki so igrali za ta klub. Naloge se bom lotil tako, da bom podatke iz spletne strani prenesel v program Microsoft Office Excel Worksheet in oblikoval tabelo ki se bo od originalne razlikovala v tem, da bom dodal par podatkov s katerimi bom kasneje lažje operiral, nekatere pa vzel ven iz originalne tabele. Za vsakega igralca bom podal:

* državo iz katere igralec prihaja ( imenska spremenljivka ).

* standardno pozicijo v igri v angleščini (forward, midfielder, winger,...).

* Teža igralca ( številska spremeljivka v kg )

* število nastopov za klub ( številska spremnljivka ).

* goli za klub ( številska spremenljivka ).

* naziv nogometaša glede na število zadetkov v klubu ( igralce bom razporedil v kategorije: legenda, strelec, povprečnež, začetnik. )

Svoje potatke bom uvozil iz spletnega medija: 


\begin{enumerate}
\item{\url{http://en.wikipedia.org/wiki/List_of_FC_Barcelona_players
}}
\end{enumerate}

\section{Obdelava, uvoz in čiščenje podatkov}
V začetku druge faze sem uvozil tabelo, ki sem jo dobil tako, da sem iz spletne strani podatke pretvoril v Excel 2013, nato pa jih shranil v CSV obliki. Tabeli sem dodal dotaten stolpec ki glede na število nastopov pove kakšno "oznako" ima igralec v klubu. Tabelo sem malo vizualno polepšal ter uporabil ukaz row.names, da sem pridobil imena vrstic. 
Sledilo je konstruiranje grafov. Najprej sem se odločil da bom izdelal grafe o golih in nastopih. Napisal sem kodo za graf v kateri sem moral zaradi preglednosti izbrisati xlab (IGRALCI) in pod tabelo dodati imena igralcev. Risal sem s funkcijo barplot() s katero sem prikazal podatke iz tabele. Prvi graf sem poimenoval "ŠTEVILO ZADETKOV IGRALCEV, KI SO ZADELI VEC KOT 100−KRAT", drugega pa "ŠTEVILO IGRALCEV Z VEČ KOT 80 NASTOPI". Za prvi graf sem izbral modro, za drugi pa rdečo barvo.


\section{Analiza in vizualizacija podatkov}



\section{Napredna analiza podatkov}


\end{document}
