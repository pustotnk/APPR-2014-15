\documentclass[11pt,a4paper]{article}
\usepackage[slovene]{babel}
\usepackage[utf8x]{inputenc}
\usepackage{graphicx}
\usepackage{url}
\usepackage{hyperref}
\usepackage{pdfpages}
\pagestyle{plain}
\begin{document}
\title{Poročilo pri predmetu \\
Analiza podatkov s programom R}
\author{Tomaž Pustotnik}
\maketitle

\section{Izbira teme}

Za tematiko moje seminarske naloge, sem si kot navijač nogometnega kluba FC Barcelona izbral analizo podatkov najboljših strelcev vseh časov, ki so igrali za ta klub. Zanimali so me predvsem splošni podatki o klubu, kateri igralci so bili v svojem času "bogovi nogometa" v klubu FC Barcelona, poleg vsega tega pa sem se ustavil še pri vprašanju, ali vsemznani nogometaš FC Barcelone Lionel Messi res izstopa toliko, kot se govori v primerjavi z ostalimi najboljšimi strelci za klub. Glede na gole za klub po letih Messija bom poiskušal napovedati koliko jih bo zabil v naslednjih letih.

Moje predvidevanje pred začetkom in analizeranjem podatkov je, da Lionel Messi v primerjavi z drugimi igralci, ne bo izstopal toliko kot bi si morda kakšen njegov oboževalec predstavljal.

Naloge se bom lotil tako, da bom podatke iz spletne strani prenesel v program Microsoft Office Excel Worksheet in oblikoval tabelo ki se bo od originalne razlikovala v tem, da bom dodal par podatkov s katerimi bom kasneje lažje operiral, nekatere pa vzel ven iz originalne tabele. Za vsakega igralca bom podal:
\itemize
\item državo iz katere igralec prihaja ( imenska spremenljivka ).

\item standardno pozicijo v igri v angleščini (forward, midfielder, winger,...).

\item Višina igralca ( v centimetrih )

\item število nastopov za klub ( številska spremnljivka ).

\item goli za klub ( številska spremenljivka ).

\item naziv nogometaša glede na število zadetkov v klubu ( igralce bom razporedil v kategorije: Legend, Almost legend, Beginner ).

Svoje potatke bom uvozil iz spletnega medija: 


\begin{enumerate}
\item{\url{http://en.wikipedia.org/wiki/List_of_FC_Barcelona_players
}}
\end{enumerate}

\section{Obdelava, uvoz in čiščenje podatkov}
V začetku druge faze sem uvozil tabelo, ki sem jo dobil tako, da sem iz spletne strani podatke pretvoril v Excel 2013, nato pa jih shranil v CSV obliki. Tabeli sem dodal dotaten stolpec ki glede na število nastopov pove kakšen "naziv" ima igralec v klubu. Igralce sem razporedil tako, da se ob igralcu z več kot 79 zadetki izpiše naziv "Legend!", ob igralcu z več kot 29 zadetki "Almost legend!", ter ob igralcu z manj kot 30 zadetki "Beginner".

Tabelo sem malo vizualno polepšal ter uporabil ukaz row.names, da sem pridobil imena vrstic. Zaradi težav pri prikazovanju imen, sem s pomočjo profesorja uvozil program ki omogoča prikazovanje posebnih znakov, tako da so se mi imena izpisala pravilno.
Sledilo je konstruiranje grafov. Najprej sem se odločil, da bom izdelal grafe o golih in nastopih ter da jih bom predstavil v tabeli. Napisal sem kodo za graf, v kateri sem moral zaradi preglednosti izbrisati xlab (IGRALCI) in pod tabelo dodati imena igralcev. Risal sem s funkcijo barplot(), s katero sem prikazal podatke iz tabele. Prvi graf sem poimenoval "ŠTEVILO ZADETKOV IGRALCEV, KI SO ZADELI VEC KOT 100−KRAT", drugega pa "ŠTEVILO NASTOPOV IGRALCEV Z VEC KOT 150 NASTOPI". Za prvi graf sem izbral modro, za drugi pa rdečo barvo. Pri prvemu grafu sem izbral las=1, da mi je imena izpisalo vodoravno, pri drugem pa ukaz las=2, da mi je imena izpisalo navpično, popravil sem velikost črk in vsa imena ki so se mi izpisovala z vprašaji tako da so na grafu razvidna cela imena.
\makebox[\textwidth][c]{
\includegraphics[width=1.35\textwidth]{../slike/grafi1.pdf}
}
\makebox[\textwidth][c]{
\includegraphics[width=1.35\textwidth]{../slike/grafi2.pdf}
}

\section{Analiza in vizualizacija podatkov}
Tretjo fazo sem začel z uvozom zemljevida ki bo prikazoval s katerih držav in kontinentov prihajajo igralci Barcelone. Na zemljevidu so razvidni naslednji kontinenti: Evropa, Afrika, Južna Amerika, iz naslednjih kontinetov so vsi moji igralci, za lepšo postavitev in prikaz pa sem dodal še Severno Ameriko. Z ukazi in malo igranja sem zemljevidu dodal imena, ter jih nato, ker so bili narobe postavljeni, z uporabo koordinat prestavil na pravo mesto. Našel sem tudi koordinate mojega stadiona, ter jih vnesel na zemljevid, na točki koordinat sedaj stoji križec ki sem ga obarval oranžno. Nekatere države sem zaradi dolžine imena skrajšal na njihove uradne kratice. Poigral sem se še z barvami, zemljevid in države, ki niso pomembne sem pobarval s črno,...

\makebox[\textwidth][c]{
\includegraphics[width=1.2\textwidth]{../slike/igralci.pdf}
}

\section{Napredna analiza podatkov}


\end{document}
